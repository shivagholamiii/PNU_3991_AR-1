\documentclass [12pt]{beamer}
\usepackage{xcolor}	
\usepackage{tikz}
\usetheme{Warsaw}
\useoutertheme{infolines}
\usepackage{ragged2e}
\begin{document}
	
\section*{kholase safahat 121 to 123}
\subsection*{shiva gholami }	
\begin{frame}
\justifying	
A number of related research techniques have been developed that gather and consolidate the opinions of expert groups, which we refer to as agreement production techniques.  The most famous of these is the Delphi method, which has been used for over fifty years to solve various types of policies, forecasts and decision problems.
The researcher sometimes provides background material or suggested information that participants can consult to better understand their situation.
\end{frame}

\begin{frame}
\justifying	
The communication structure process represents a concept that relates conflicting views to the data and purpose of the research project.  Opposing views are not considered to be anti-objectivity.  Rather, opposing beliefs actually serve objectivity.  Therefore, this method may not lead to convergence of opinions.
Consensus techniques can serve to gather and 

articulate collective wisdom, as well as serve a coherent function and community, bring together diverse ideas, and allow individuals to work as an effective group.
\end{frame}

\begin{frame}
\justifying	
Some types of consensus groups have evolved through face-to-face meetings, and recently the immediate conclusion of the results of their decisions using computers and software systems known as decision software has been given to these groups.  it helps.  .  To overcome the time and space constraints and costs associated with face-to-face meetings, consensus techniques also use e-mail or courier services, which allow members to post and defend their reasons without face-to-face consultation.
\end{frame}
\end{document}
