{\documentclass [10pt,a4paper,]{book}
	
\begin{document}
		
\begin{flushleft}
\textbf{122} CHAPTER NINE   
\end{flushleft}
the IQR and median or bimodal distributions) are also noted and shared with the group to ensure that critical opposing opinions are not ignored. The process thereby dictates a conceptual communication structure that relates the opposing opinions to the data and objective of the research project. Opposing opinions are not considered to be antithetical with objectivity; rather, opposing opinions actually serve objectivity. This technique, then. may not lead to a convergence of opinions; bimodal distributions will always remain a possible outcome. However, the resulting outcome, irrespective of whether or not an opinion synthesis occurs, may lie more valid than other methodolo-gies because of the acknowledgment and accommodation of opposing opinions. Through the progress of successive iterations, the consensus process works to evolve an informed and well-thought-out answer to a difficult question. The question is often of such complexity or deals with future forecasts for which there is no way to calculate a single correct answer. As such, consensus techniques can serve to gather and articu-late communal wisdom, as well as serve a cohesive and community function, bringing diverse opinions together and allowing individuals to work as an effective group. 


The consensus process benefits if participants are able to view their opinions in relationship to those of the rest of the group and are given an opportunity and motiva-tion to argue and defend their opinions. The resulting dialogue allows individuals (and the group) to alter and refine their opinions, hopefully leading to an informed and wise consensus of all participants. On the other hand, there is a danger that the consensus process will rapture only collective ignorance. However, the selection of informed and motivated participants coupled with clear goals, objectives, and processes, usually results in consensus agreements that gather and expand the wisdom of all members. 


Consensus-building theory has evolved into a series of techniques known as con-sensus research with three methodological variations or processes—Delphi Method, Nominal Groups Technique, and Consensus Development Conference (Murphy et al., 1998). The Net provides ways to not only expand, but also improve, both the effec-tiveness and efficiency of these traditional forms of consensus research.

\begin{flushleft} 
\textbf{ADVANTAGES OF CONSENSUS TECHNIQUES}
\end{flushleft} 
Consensus techniques for c-research arc related to focus groups and offer many of the same benefits and challenges. However, they provide more formal structure than focus group discussions. They arc more deliberately focused on achieving a single best answer or statistically revealing the extent of disagreement than the open-ended and qualitative nature of most focus group research. Moreover, consensus groups provide a number of useful advantages for e-researchers seeking a means to gather knowledge from a dispersed group of experts.

\begin{flushleft} 
\textbf{Advantages of Traditional Consensus Techniques} 
\end{flushleft}
\textbf{High-Quality and Informed Opinion.} Consensus groups are usually purposively selected so that the participants arc informed, interested, and capable of providing high-quality opinions. Participants in consensus groups draw first on their own experiences 
\end{document}			
