{\documentclass [10pt,a4paper,]{book}
	
\begin{document}
		
\begin{flushright}
NET- BASED CONSENSUS TECHNIQUES  \textbf{121}
\end{flushright}
\begin{center}
\begin{itemize}
\item 
The research problem does not lend itself to well-defined systematic techniques; it can, however, acquire useful results from subjective judgments on a collective basis. 

\item 
The research participants will be representative of diverse backgrounds, with respect to experience and expertise, and are geographically dispersed, making frequent group meetings virtually impossible due to time and cost.

\item 
Related to the prior point (that the participants will have diverse backgrounds), the experts may be uncompromising between opinions in a may that the com-munication process must be refereed and/or anonymity assured.

\item 
The heterogeneity of the research participants must be preserved to avoid dom-ination by some experts.
\end{itemize}
\end{center}
A number of related research techniques have been developed that gather and con-dense opinions from groups of experts, which we refer to as consensus-generating techniques. The most well known of these is the Delphi Method, which has been used for over fifty years to resolve a wide variety of policy, forecasting, and decision-making problems. Consensus techniques have mostly been used in fare-to-face conference modes; however, we are seeing increased interest in their use at a distance using a vari-ety of communications technologies. The Net, with its capacity to support a variety of synchronous and asynchronous, as well as group and individual communica-tions modes, is an ideal environment to support existing and experiment with new varieties of consensus data collection. In this chapter we oveniew the ways in which these techniques have been used and discuss ways the techniques can be adapted for Net use.


Generally, consensus techniques work by soliciting the opinion(s) of experts (usually in some sort of individual format) on an important issue or question. The researcher sometimes provides background materials or suggested information refer-ences that participants can consult to better inform their position. The range of the group responses, along with their individual opinions, are then returned to the partic-ipant. The participant is asked to defend, explain, or change their opinions so as to move the group to a single-best answer or consensus. This process may be repeated two or more times and hopefully as individual differences are reduced, a consensus develops. An important feature of the process is the facilitation and encouragement of individuals to share the rationale for their opinions—especially if these differ Irian the group mean. This sharing is facilitated through distribution of the written rationale or comments to the survey questions (using postal services, fax, email, or Web) or through a structured discussion during a face-to-face meeting or real-time distributed meeting or conference. 


The objective of the consensus process is to arrive at a single statement or answer that participants can agree on. Failing this unanimity, the process should clearly iden-tify the nature and extent of opinion divergence. Generally consensus is sought; how-ever, we are mindful of Mahatma Gandhi's observation that "honest disagreement is often a good sign of progress." A measurement of central tendency, the interquartile range (IQR), or the simple mean is usually used to determine the consensus of opinion for each of the questions posed to the group. Opposing opinions (those who fall outside 
\end{document}			
		