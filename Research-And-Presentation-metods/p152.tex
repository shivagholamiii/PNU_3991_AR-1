\documentclass [10pt,a4paper]{book}

\begin{document}
\begin{flushleft}
\textbf{152   CHAPTER ELEVEN}
\end{flushleft}
\begin{flushleft}
\textbf{Reducing e-Survey Error}
\end{flushleft}
Every survey is subject to error. Even if one were able to survey each member of the
target population, there may still be some error due to respondent misinterpretation of
the questions or misrepresentation of themselves.However, since surveying all mem-
bers is rarely possible, additional errors due to sample selection may also occur.
Despite great care in selecting a sample, there will always be random variations in any
population that may, quite by chance, bias even the most meticulously designed and
administered survey. However, it is the responsibility of researchers to eliminate as
much of the systemic error in their design and administration of the e-survey as possi-
ble, Next we describe the major sources of error, common to all forms of survey, with
brief notations on the particular manifestation of the error in Net-based forms of sur-
vey research. Major sources of error reduce the value, veracity, and impact of any
survey—including those conducted online.
\begin{flushleft}
\textbf{Frame or Coverage Error}
\end{flushleft}
Coverage error occurs when only a particular subset of the target population is
included in the survey. The sampling frame is the list or source of names from which
the sample is drawn. If this list does not contain all of the members of the population,
and especially if some groups or individuals are systematically eliminated from the
frame, then frame error will result in survey result errors. This is an obvious danger for
¢-researchers, in that the entire general population does not currently have access to
the Net. Thus for the foreseeable future there will always be elements of the whole
population that are eliminated from a Net-based survey due to coverage error. How-
ever, there are a growing number of target populations to whom 100 percent or close
to 100 percent of the members are online. This group would include employees of
many companies and members of certain professions or social organizations. Coverage
error has led some researchers to conclude that e-surveys are not useful (Dillman,
2000) for general population studies at this time. Although we agree that one cannot
make inferences about the whole population based on the subset who use the Net, we
contend that there is still a great deal of valuable information that can be obtained from
sampling from the growing number of people who access the Net on a regular basis.
\begin{flushleft}
\textbf{Measurement Error}
\end{flushleft}
Measurement error occurs when there is a variation between the information the
researcher is looking for and that obtained from the research process. Measurement
error can begin in the design process if the researcher is not clear what type of infor-
mation is being sought. It is most commonly found in measurement bias within the
survey itself, in the form of confusing, uninterpretable, or biased questions producing
results that are inaccurate, uninterpretable, or both. Measurement error may also occur
during completion of the survey if respondents make data entry errors when complet-
ing the survey. Finally, measurement error may result from error in data analysis.
Careful wording of instructions and provision of examples are useful ways to reduce
measurement error.

\end{document}
