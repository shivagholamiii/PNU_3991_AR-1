{\documentclass [10pt,a4paper,]{book}
	
\begin{document}
		
\begin{flushright}
NET- BASED CONSENSUS TECHNIQUES  \textbf{123}
\end{flushright}
and opinions, and then build on that knowledge by considering the opinions and exper-tise of others. This creates an environment for social cognition that is likely to produce better decisions than those made by individuals and to arrive at negotiated consensus from expert opinions. As Langford (1972) notes. the consensus data gathering technique endeavors to make "effective use of informed intuitive judgment; it is designed to com-bine individual judgments systematically and thus obtain a reasoned consensus" (p. 21). 
\begin{flushleft} 
\textbf{Safety in Numbers}. Consensus groups are less likely to arrive at or support incor-rect answers or ineffective solutions because they are working with the collective expertise of a number of experts with a variety of experiences.
\end{flushleft}
\begin{flushleft} 
\textbf{Authority.} Group decisions are more likely to be taken seriously than those of any individual. In addition, specific consensus techniques have been shown to be more reli-able and valid than other forms of opinion gathering and synthesis. 
\end{flushleft}
\begin{flushleft} 
\textbf{Controlled Process.} Consensus techniques provide a set of procedures that tend to mitigate the negative impacts of group behavior, such as coercion, domination by cer-tain individuals, or premature consensus seeking. A structured process can eliminate these kinds of group behavior.
\end{flushleft}
\begin{flushleft} 
\textbf{Supports Communication among Individuals with Polarized Views.} Although consensus techniques may not always result in individuals coming to a unified position, they do create an environment in which polarized views can be democratically expressed and negotiated with equity. In some applications, anonymity is used to allow participants to freely state and argue their positions without threat of retaliation.
\end{flushleft}
\begin{flushleft} 
\textbf{Credibility.} Although these techniques are not without their technical critics, vari-ous mathematical techniques can be applied at each stage of the process to quantify individual and group opinions. The feedback of results to individuals allows partici-pants to judge their opinions in relationship to the larger group. At the end of the process the extent of consensus can be accurately calculated and discussed.
\end{flushleft}
\begin{flushleft} 
\textbf{Accessibility.} Some types of consensus groups have evolved through the use of face-to-face meetings, and more recently these groups have been aided by the instant com-putation of the results of their decision using computers and software systems generally known as decision-making software. To overcome the time and space restrictions and costs related to face-to-face meetings, consensus techniques have also used postal mail or courier services to allow members to post and defend their reasons without meet-ing face-to-face. However, the inherent time delay, inconvenience, and cost of postal returns remain problematic.
\end{flushleft}
\begin{flushleft}  
\textbf{Additional Advantages}
To these advantages of traditional consensus techniques, a number of others can be added when the Net is used as the means of communication.
\end{flushleft} 
\end{document}
